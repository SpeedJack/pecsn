\chapter{Data analysis}\label{ch:analysis}

To perform the analysis of the three scenarios we have prepared a shell
script\footnote{\code{simulations/simulate.sh}.} that executes parallel runs
with the configuration specified by a command line parameter and exports the
results inside a specified folder in the \code{analysis/} directory in CSV
format.

The \code{analysis/} directory contains various Jupyter notebook used to analyze
the data. Those notebooks have been developed with automation in mind: we just
need to configure it in the ``Config'' section at the start of the notebook and
run it. Python will then output a complete analysis on the provided data. This
means that, often, not all the plots/tables created by these scripts are really
useful for the analysis (they are plotted, but this does not mean they are
meaningful in every situation). Moreover, when we need to run the same Jupyter
notebook with different configuration parameters \exgratia{to apply or not a
transformation on the predicted variable in the 2\textsuperscript{k}r assumption
verification} we will just copy and paste the notebook and change the needed
configuration parameters.

The \code{analysis/} folder contains ``generic'' Jupyter notebooks that we will
copy inside some other folder before run it for a specific scenario. So, for
example, analysis for the ``High density scenario'' can be found inside the
\code{analysis/HighDensity} folder.

For each Jupyter notebook execution we have also saved the output in HTML
format. So, as an example, for the ``High density scenario'', inside the
\code{analysis/HighDensity/exported\_html/} you can found the output of each
Jupyter notebook of the \code{analysis/HighDensity/} directory. This may be
useful to inspect the results of our analysis even if Jupyter is not installed.

In the following we will discuss the most important considerations on the
results provided by Python. In any case, we will ofter refer to the Jupyter
notebook file which contains the full analysis. So, for example, if in the
``High density scenario'' we mention the \code{2kr.ipynb} file, head to the
\code{analysis/HighDensity/2kr.ipynb} file to read the full analysis (or the
equivalent HTML exported version inside the \code{exported\_html/} sub-folder).

\section{High density scenario}\label{sec:high-density}

\section{2\texorpdfstring{\textsuperscript{k}}{k}r analysis}\label{sec:2kr}


\section{Low density scenario}\label{sec:low-density}

\section{2\texorpdfstring{\textsuperscript{k}}{k}r analysis}\label{sec:2kr}

\subsection{Testing the assumptions}\label{subsec:hdassumptions}

The assumptions of normality, independency and finite variance must be verified
for the residuals of the observations. The complete analysis can be found in
\code{2kr-assumptions-tests.ipynb}.

\subsubsection{Normality}\label{subsubsec:ldassumptionsnormality}

In this case, we get better results for normality compared to those obtained in
the high density scenario: also the QQ-plot for the total number of messages
sent looks more linear with \(R^2\!=\!0.9322\).

Also in this case normality is not verified for the percentage of covered users.
Since we do not study this index, we do not need to get it verified.

\subsubsection{Independency}\label{subsubsec:hdassumptionsindependency}

Independency is assumed by the way we have conducted the experiments. Anyway,
the scatter plots do not show any trend.

\subsubsection{Finite variance}\label{ldassumptionsvariance}

As in the case of high density, the scatter plots for the variance of the
residuals of the total number of collisions and the total number of messages
sent don't show trends, only a step in the right of the message plot (but with
residuals one order of magnitude less than the predicted response). So we can
conclude that the assumption of finite variance is still valid.

As for the high density, to verify the assumption for the residuals of the 99th
percentile of the total broadcast time we need to perform a logarithmic
transformation of the predicted variable. The result is shown in
\figref{fig:ldtimevariance}.

\begin{figure}[htb]
	\centering
	\includegraphics[width=0.37\textwidth]{img/ld/broadcasttime-variance-transform}
	\caption{Scatter plot of the variance of the residuals of the total
	broadcast time. No trend is shown after the logarithmic
	transformation}\label{fig:ldtimevariance}
\end{figure}

So, in order to meet the requirement of finite variance for the residuals of the
total broadcast time, a transformed model is needed: \(y' = \ln(y)\).



\section{Rectangular scenario}\label{sec:rectangular}

\section{2\texorpdfstring{\textsuperscript{k}}{k}r analysis}\label{sec:2kr}


