\subsubsection{Broadcast time}\label{subsubsec:ld2krtime}

Since in all the experiments we have always reached a coverage of at least
\(99\%\), here we will only discuss the broadcast time needed to reach the 99th
percentile of the coverage.

We have a large unexplained variation (\(9.08\%\)), as in the case of the high
density scenario. The reason for this result is the same and it will be
discussed in \chref{ch:starting-node}.

We can see that the most important factor is the broadcast radius that accounts
for the \(58.90\%\) of the variation, followed by the size of the hear window
(\(30.43\%\)). Other factors and their combinations are irrelevant. These
results are aligned with those obtained for the high density scenario, so the
same considerations apply.

Also in this case we had to use a logarithmic transformation of the predicted
variable in order to meet the assumption of finite variance for the residuals.
