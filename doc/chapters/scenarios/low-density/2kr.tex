\subsection{2\texorpdfstring{\textsuperscript{k}}{k}r analysis}\label{subsec:ld2kr}

Analysis performed with \(k\!=\!4\) and \(r\!=\!10\), for a total of \(2^4 \cdot
10 \!=\!160\) experiments. The performance indexes evaluated are the one defined
in \secref{sec:indexes}. When we talk about the total number of messages sent
remember that we are analyzing the \emph{energy efficiency} (\(\mathit{Eff}\))
of the network.

The simulation configuration for this analysis, named ``LowDensity2kr'' can be
found in the \code{simulations.ini} file. The simulations have been run using
our \code{simulate.sh} script with the following command:
\begin{verbatim}
$ ./simulate.sh -s LowDensity -c LowDensity2kr
\end{verbatim}

First, we have verified that the range of values for the \code{maxCopies}
parameter (2--6) was fine. The analysis can be found in \code{histograms.ipynb}.
Here, we just check that with \(m\!=\!2\) and \(m\!=\!6\) we get that a lot of
users decide to not relay the message in the first case and only a small bunch
of users decide to not relay the message in the second case.

Then, the 2\textsuperscript{k}r analysis can be found in \code{2kr.ipynb}. We
will verify the assumptions of normality, independence and finite variance for
the residuals in \secref{subsec:ldassumptions}. Here we will discuss the
results.

\subsubsection{Percentage of covered users}\label{subsubsec:rect2krcoverage}

\ldots

\subsubsection{Total number of collisions}\label{subsubsec:rect2krcollisions}

For the total number of collisions we get a very low unexplained variation
(\(0.72\%\)). The broadcast radius accounts for the \(64.31\%\) of the variation
and it is the dominant factor, as in other scenarios. As in the low density
scenario and differently from the high density one, the second most important
factor is the maximum number of copies, which accounts for the \(16.47\%\) of
the variation while the maximum relay delay (third factor) accounts only for the
\(4.86\%\) of the variation. In this case, also the combination of the broadcast
radius and the maximum number of copies is relevant (\(9.59\%\)). This results
are quite surprisingly because the density of the rectangular scenario is the
same as the high density one (which is a square).

As in the case of other configurations, we need to decrease the broadcast radius
to reduce the total number of collision, which is an obvious consideration, and
as shown in \figref{fig:rectperfcollisionsm} using lower values for the maximum
number of copies reduces the total number of collisions.

The fact that the broadcast radius is less important in this scenario compared
to the high density case, but more important with respect to the low density,
can be explained by the fact that the users are scattered in the floorplan, so
we need a huger increase of the broadcast radius to see an effect, but they are
not so well distributed because they are flattened in the rectangle.

\begin{figure}[hbt]
	\centering
	\includegraphics[width=0.5\textwidth]{img/rect/collisions_m_perfplot.png}
	\caption{Decrease the maximum number of copies to decrease the total
	number of collisions}\label{fig:rectperfcollisionsm}
\end{figure}

\subsubsection{Total number of messages sent}\label{subsubsec:ld2krmessages}

This is an indication of the energy efficiency of the entire network.

The maximum number of copies is the dominant factor (\(85.18\%\)), as in the
high density case. In the previous case the size of the hear window was the
second most important factor. Here, instead, we have the broadcast radius
(\(6.57\%\)) and its combination with the maximum number of copies. The
unexplained variation is extremely low \(0.31\%\).

From \figref{fig:ldperfmessagesR} we notice that increasing the broadcast radius
is useful to reduce the total number of messages sent when the maximum number of
copies is low. This can be explained by the fact that with an higher value for
\(m\) means that nearly all the users in the network will relay the message
after they have successfully heard it. So, if \(m\) is high, there is not much
to do to reduce the total number of messages sent.

\begin{figure}[htb]
	\centering
	\includegraphics[width=0.6\textwidth]{img/ld/messages-R-perfplot}
	\caption{Performance plot for the total number of messages sent. The
	broadcast radius can be used to reduce the total number of messages sent
	when the maximum number of copies is not
	high}\label{fig:ldperfmessagesR}
\end{figure}

\subsubsection{Broadcast time}\label{subsubsec:hd2krtime}

Since in all the experiments we have always reached a coverage of at least
\(99\%\), here we will only discuss the broadcast time needed to reach the 99th
percentile of the coverage.

We have a large unexplained variation (\(7.95\%\)). The reason for this result
will be discussed in \chref{ch:starting-node}.

We can see that the most important factor is the broadcast radius that accounts
for the \(71.25\%\) of the variation, followed by the size of the hear window
(\(19.30\%\)). Other factors and their combinations are irrelevant. We note that
these variations are much larger than the unexplained variation, so we can still
say that they are significant. Of course, their \(95\%\) confidence intervals
also gets larger: (\(63.62\%\), \(79.30\%\)) for \(R\) and (\(15.43\%\),
\(23.59\%\)) for \(T\), but they do not include the zero.

This means that we can reduce the broadcast time by increasing the broadcast
radius to let the relayed messages to reach more user. Also the size of the hear
window can be decreased in order to reduce the time that each user wait before
deciding to relay or not relay the message. Of course, as stated before,
reducing the size of the hear window has a negative impact on the energy
efficiency, so some trade-off considerations are required.

We note that we have performed a \emph{logarithmic transformation of the
predicted variable} \idest{the total broadcast time} in order to meet the
assumption of finite variance for the residuals, as discussed in
\secref{subsec:hdassumptions}. This means that the final approximated regression
model for the broadcast time is transformed into the one shown
in~\eqref{eq:hdtimelogregressionmodel}, with \(R\) and \(T\) normalized between
\(-1\) and \(1\). (factors with low influence are removed; 95\% confidence
intervals are show in parenthesis).

\begin{equation}\label{eq:hdtimelogregressionmodel}
	\begin{cases}
		\log(t_B) = q_0 + q_R \cdot R + q_T \cdot T + e\\
		q_0 = 4.782874\\
		q_R = -0.462452 & (-0.487897, -0.437006)\\
		q_T = 0.240663 & (0.215217, 0.266109)
	\end{cases}
\end{equation}

We can then predict the total broadcast time using the formula shown
in~\eqref{eq:hdtimeregressionmodel}.

\begin{equation}\label{eq:hdtimeregressionmodel}
	t_B = e^{4.782874 - 0.462452 \cdot R + 0.240663 \cdot T}
\end{equation}

And the 95\% confidence interval as shown in~\eqref{eq:hdtimeregressionci}.
(TODO\@: not sure it is right).

\begin{equation}\label{eq:hdtimeregressionci}
	\begin{cases}
		t^-_B = e^{4.782874 - 0.487897 \cdot R + 0.215217 \cdot T}\\
		t^+_B = e^{4.782874 - 0.437006 \cdot R + 0.266109 \cdot T}\\
	\end{cases}
\end{equation}

