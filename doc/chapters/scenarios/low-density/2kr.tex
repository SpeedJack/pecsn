\subsection{2\texorpdfstring{\textsuperscript{k}}{k}r analysis}\label{subsec:ld2kr}

The ``low density scenario'' is characterized by a square floor (200m X 200m),
covered by 200 users, placed in a random position, different from every
experiment.  For each possible configuration (16), in the 2kr analysis we did
ten experiments. This type of analysis studies the number of collisions, the
number of messages sent and the broadcast time to reach some coverage
percentiles and for each measure tries to find the influence of each factor in
the results.

\subsubsection{Number of collisions}

The results of the analysis for the number of collisions, in terms of variation
percentages are:
\begin{itemize}
	\item \textbf{Hear Window:} \(0,04\%\)
	\item \textbf{Max Relay Delay:} \(5,44\%\)
	\item \textbf{Max Copies:} \(1,3\%\)
	\item \textbf{Broadcast Radius:} \(87,48\%\)
	\item \textbf{Unexplained Variation:} \(3,88\%\)
\end{itemize}

We can see, with a very low unexplained variation, that the dominant factor is
the broadcast radius, with also a little influence of the maximum relay delay.
No one of the not indicated combined factors account for more than \(1\%\).

\subsubsection{Number of messages}

The analysis for the number of collisions, in terms of variation percentages,
has given the following results:
\begin{itemize}
	\item \textbf{Hear Window:} \(2,03\%\)
	\item \textbf{Max Relay Delay:} \(12,15\%\)
	\item \textbf{Max Copies:} \(48,37\%\)
	\item \textbf{Broadcast Radius:} \(2,83\%\)
	\item \textbf{Unexplained Variation:} \(12,34\%\)
\end{itemize}

In this case the dominant factor is the maximum number of copies, with also a
not negligible influence of the maximum relay delay. For the combined factors,
we have that the combination of max relay delay and max copies account for
\(12,77\%\) of the total variation. The other combinations are negligible.  In
this case the unexplained variation is higher, but the complementary coefficient
of determination is higher than \(85\%\)

\subsubsection{Broadcast Time}

The analysis for the broadcast time is done with different percentiles and we
considered a lognormal transformation instead of the linear one:

\begin{description}
	\item [$\textbf 25^{th}$ percentile] of the coverage:
		\begin{itemize}
			\item \textbf{Hear Window:} \(20,82\%\)
			\item \textbf{Max Relay Delay:} \(0,04\%\)
			\item \textbf{Max Copies:} \(0,93\%\)
			\item \textbf{Broadcast Radius:} \(50,03\%\)
			\item \textbf{Unexplained Variation:} \(26,14\%\)
		\end{itemize}
	\item [$\textbf 50^{th}$ percentile] of the coverage:
		\begin{itemize}
			\item \textbf{Hear Window:} \(23,54\%\)
			\item \textbf{Max Relay Delay:} \(0,13\%\)
			\item \textbf{Max Copies:} \(0,43\%\)
			\item \textbf{Broadcast Radius:} \(55,4\%\)
			\item \textbf{Unexplained Variation:} \(17\%\)
		\end{itemize}
	\item [$\textbf 75^{th}$ percentile] of the coverage:
		\begin{itemize}
			\item \textbf{Hear Window:} \(26,39\%\)
			\item \textbf{Max Relay Delay:} \(0,01\%\)
			\item \textbf{Max Copies:} \(0,26\%\)
			\item \textbf{Broadcast Radius:} \(55,77\%\)
			\item \textbf{Unexplained Variation:} \(16,08\%\)
		\end{itemize}
	\item [$\textbf 90^{th}$ percentile] of the coverage:
		\begin{itemize}
			\item \textbf{Hear Window:} \(28,76\%\)
			\item \textbf{Max Relay Delay:} \(0,03\%\)
			\item \textbf{Max Copies:} \(0,15\%\)
			\item \textbf{Broadcast Radius:} \(54,82\%\)
			\item \textbf{Unexplained Variation:} \(15,01\%\)
		\end{itemize}
	\item [$\textbf 95^{th}$ percentile] of the coverage:
		\begin{itemize}
			\item \textbf{Hear Window:} \(30,36\%\)
			\item \textbf{Max Relay Delay:} \(0\%\)
			\item \textbf{Max Copies:} \(0,28\%\)
			\item \textbf{Broadcast Radius:} \(54,57\%\)
			\item \textbf{Unexplained Variation:} \(13,34\%\)
		\end{itemize}
	\item [$\textbf 99^{th}$ percentile] of the coverage cannot be studied
		because some experiments don't reach the value
\end{description}

In all the cases the dominant factors are broadcast radius (the highest) and
hear window. Note that the precision is very low for low percentiles (so for
these the analysis cannot be considered), but for high percentiles we are around
the \(85\%\) of the determination coefficient.

\subsubsection{Coverage}

We tried to do the analysis also for the coverage but for this parameter the
model is very inaccurate (around \(80\%\) of unexplained variation), so it
cannot be considered.

\begin{itemize}
	\item \textbf{Hear Window:} \(0\%\)
	\item \textbf{Max Relay Delay:} \(8,85\%\)
	\item \textbf{Max Copies:} \(0,77\%\)
	\item \textbf{Broadcast Radius:} \(0,77\%\)
	\item \textbf{Unexplained Variation:} \(82,15\%\)
\end{itemize}
