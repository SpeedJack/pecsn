\subsubsection{Broadcast time}\label{subsubsec:hd2krtime}

Since in all the experiments we have always reached a coverage of at least
\(99\%\), here we will only discuss the broadcast time needed to reach the 99th
percentile of the coverage.

We have a large unexplained variation (\(8.03\%\)). The reason for this result
will be discussed in \chref{ch:starting-node}.

We can see that the most important factor is the broadcast radius that accounts
for the \(67.88\%\) of the variation, followed by the size of the hear window
(\(22.82\%\)). Other factors and their combinations are irrelevant. We note that
these variations are much larger than the unexplained variation, so we can still
say that they are significant. Of course, their \(95\%\) confidence intervals
also gets larger: (\(60.41\%\), \(75.79\%\)) for \(R\) and (\(18.58\%\),
\(27.5\%\)) for \(T\), but they do not include the zero.

This means that we can reduce the broadcast time by increasing the broadcast
radius to let the relayed messages to reach more user. Also the size of the hear
window can be decreased in order to reduce the time that each user wait before
deciding to relay or not relay the message. Of course, as stated before,
reducing the size of the hear window has a negative impact on the energy
efficiency, so some trade-off considerations are required.

We note that we have performed a \emph{logarithmic transformation of the
predicted variable} \idest{the total broadcast time} in order to meet the
assumption of finite variance for the residuals, as discussed in
\secref{subsec:hdassumptions}. This means that \ldots TODO \ldots.
