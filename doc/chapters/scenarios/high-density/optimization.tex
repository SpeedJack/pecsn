\subsection{Optimization for the high density
scenario}\label{subsec:hdoptimization}

As we have seen, there is not much to do to optimize the percentage of covered
users, since it is always nearly perfect. But, of course, there are very low
values of the parameters that does not allow the network to reach the maximum
coverage. We have run a full factorial analysis, with low values of each
parameter, in order to identify the minimum values required to reach the
coverage. Each possible configuration (\(750\)) has been run with 10
repetitions. Figure X shows the \ldots. The complete analysis can be found in
\code{coverage.ipynb}. The configuration used is named
``HighDensityCoverage''.

TODO

If we want to optimize the total broadcast time, as we have seen in
\secref{subsec:hd2kr}, we need to increase the broadcast radius or decrease the
size of the hear window. File \code{broadcast-time.ipynb} shows the results with
various combination of these two factors. The configuration used is named
``HighDensityTime''. Figure Y shows the \ldots.

TODO

If instead we want to optimize the energy efficiency, we must reduce the
broadcast radius and the total number of messages sent. In file
\code{messages.ipynb} we have performed a factorial analysis with low values for
the broadcast radius. Factors taken into considerations are the one we have
identified as relevant in \secref{subsec:hd2kr}, namely the maximum number of
copies and the size of the hear window.

TODO
