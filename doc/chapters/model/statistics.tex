\section{Collected statistics}\label{sec:statistics}

All \code{@statistic} are defined in the \code{Floorplan} network's NED file and
they are collected using the \omnetpp{} signaling system by the \code{Oracle}
and the \code{User} modules.

The following statistics are collected by the \code{Oracle} module and are
related to the entire \code{Floorplan} network:
\begin{description}
	\item[activityTime] When a user perform some ``activity'' \idest{receive
		or send a message, excluding self messages} the Oracle signals
		the simulation time of the event. The last value signaled is
		recorded and represents the total broadcast time (regardless of
		whether all users have been reached or not).
	\item[coveredUsers] The number of users infected \idest{those who
		successfully heard the message}. A value with timestamp is
		recorded in each slot along with the sum of all values (the
		total number of infected users at the end of the simulation).
	\item[rcvMsgsPerSlot] For each slot, the number of messages that have
		been successfully heard by users. Also general statistics
		\idest{mean, stddev, min, max, \etc} are recorded.
	\item[sntMsgsPerSlot] For each slot, the number of users who sent out
		the message. General statistics are also recorded.
\end{description}

The following statistics are collecte by the \code{User} module and are mainly
related to the single user in the simulation:
\begin{description}
	\item[collisions] The number of collision registered by the user.
		General statistics are collected along with the sum of all
		collisions in the entire network.
	\item[copies] The number of copies of the message successfully heard by
		the user during the time window between the reception of the
		message and the time at which the user relays it. General
		statistics are collected.
	\item[reachedUsers] When a user relays the message, it signals the
		number of users reached (regardless of whether they correctly
		hear the message or not). General statistics are collected.
\end{description}
