\section{Parameters}\label{sec:parameters}

\begin{itemize}
	\item \standout{Floorplan}:
		\begin{description}
			\item[userCount] \textit{(int)} The number \(N\) of
				users inside the network (default: \code{60}).
			\item[sizeX] \textit{(meters)} The horizontal size of
				the floorplan (default: \code{200m}).
			\item[sizeY] \textit{(meters)} The vertical size of the
				floorplan (default: \code{200m}).
			\item[indexStartingNode] \textit{(int distribution)} A
				random number generated from a distribution to
				select a random user to broadcast the message at
				the start of the simulation (default:
				\code{intuniform(0, userCount-1)}).
		\end{description}
	\item \standout{User}:
		\begin{description}
			\item[posX] \textit{(meters)} The X-axis position of the
				user in the floorplan (default: set by Floorplan
				as \code{uniform(0m, sizeX)}).
			\item[posY] \textit{(meters)} The Y-axis position of the
				user in the floorplan (default: set by Floorplan
				as \code{uniform(0m, sizeY)}).
			\item[sendOnStart] \textit{(bool)} Specifies if the user
				should send out the message at the start of
				simulation (default: \code{false}).
			\item[slotDuration] \textit{(seconds)} The duration of a
				time slot (default: \code{1s}).
			\item[broadcastRadius] \textit{(meters)} The broadcast
				radius \(R\) (default: \code{40m}).
			\item[hearWindow] \textit{(int)} The time window of
				\(T\) slots that the user should wait before
				relaying the message (default: \code{5}).
			\item[maxCopies] \textit{(int)} The maximum number of
				copies \(m\) that the user can receive to
				decide to relay the message (default: \code{3}).
				Note that our model counts also the first
				message as a ``copy'', but this is not an issue
				since the problem defined in \chref{ch:specs}
				says ``less than \(m\) copies'' (\emph{strict}
				constraint) while here we talk about the
				``maximum number of copies (including first
				message)'' (\emph{loose} constraint).
			\item[relayDelay] \textit{(int distribution)} The number
				of time slots \(\delta\) to add to the time
				window \(T\) before relaying the message
				(default: \code{intuniform(0, 3)}).
		\end{description}
	\item \standout{Oracle}:
		\begin{description}
			\item[slotDuration] \textit{(seconds)} The duration of a
				time slot. It should be equal to
				\code{User.slotDuration}.
			\item[timeout] \textit{(int)} The number of time slots
				with no network activity the oracle must wait
				before stopping the simulation. It should be
				\(\geq T+\max(\delta)\).
		\end{description}
\end{itemize}

\begin{tcolorbox}[title=Note]
	Since users only operates at time slot intervals, the parameter
	\code{slotDuration} does not affect the behavior and the performance of
	the network in any way. The only effect is that times \exgratia{total
	broadcast time} are scaled, so we will fix this parameter to \(1s\). In
	this way, each second of simulation time represents a time slot and the
	analysis become easier since, by default, \omnetpp{} records emitted
	signals with the timestamp at with emission occur.

	Of course this is not the case in a real system (probably each slot is
	much more shorter than 1 second) but this decision will not break any
	consideration made by our analysis, except for the time scaling effect.

	In the following we may use the terms ``1 slot'' and ``1 second''
	interchangeably.
\end{tcolorbox}
