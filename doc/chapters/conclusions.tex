\chapter{Conclusions}\label{ch:conclusions}

Optimizing a network like this is not a trivial task. A trade-off decision is
needed between the broadcast time and the energy efficiency in order to select
proper values for the size of the hear window and the broadcast radius.

The network is almost always able to reach a full coverage if the broadcast
radius is not too low. Sometimes a catastrophic event may occur, where the
network is able to reach little or no coverage due to the random distribution of
the users. This problem can be reduced by increasing the broadcast radius but it
can not be eliminated (unless, of course, a broadcast radius that allows the
first user to reach instantly the entire floorplan is used).

\emph{Trickle relaying} is very effective in reducing the number of collisions:
using a low value for the maximum number of copies leads to a very low number of
collisions.

The maximum relay delay (\(\max(\delta)\)), introduced to fix the issue
described in the note in \chref{ch:specs}, also served the purpose of reducing
the number of collisions.

Generally, the shape of the floorplan is irrelevant: the only effect is an
higher broadcast radius if the floorplan is a rectangle compared to the case
where the floorplan is a square.

The position of the first user that sends out the message is \emph{very}
important. As per specifications, the starting user is ``random'', so it's
placement in the center can not be assured. This leads to possibly higher
broadcast time, higher variance in the results, higher probability of a
catastrophe in the coverage. This issue can be address by increasing the
broadcast radius but, if it is possible, it is better to avoid the borders and
corners of the floorplan for the starting user.
