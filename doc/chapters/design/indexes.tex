\section{Performance indexes}\label{sec:indexes}

We have defined the following performance indexes:
\begin{description}
	\item[Total broadcast time] (\(t_B\)) The time required in order to
		deliver the message to all reachable users.
	\item[Percentage of covered users] (\(\mathit{\%U}\)) The number of
		users infected.
	\item[Energy efficiency] (\(\mathit{Eff}\)) We define the energy
		efficiency as an index that decreases with the increase of the
		broadcast radius \(R\) and the total number of transmissions
		\(M\). We will not develop a well-defined mathematical formula
		for the energy efficiency of wireless communications; we will
		just state that \(\mathit{Eff} \propto \frac{1}{R \cdot M}\). Of
		corse, this index is of great interest in wireless ad-hoc
		networks where each communicating device must optimize the
		energy used in order to avoid frequent battery recharges or
		replacements. Moreover, we note that from the point of view of a
		single user of our model, where the user must relay the message
		at most one time only, the only factor affecting this index is
		\(R\). So our network should try to minimize \(R\) in order to
		maximize the energy efficiency. We may consider the total number
		of transmissions only to evaluate the efficiency of the
		\emph{entire} network.
	\item[Total number of collisions] (\(C\)) The total number of
		collisions. This is of course highly related to the number of
		users in the network. We must consider that a network with an
		higher user density may have an higher number of collisions. So
		this index may not be user to compare the performance of
		networks with different number of users (at least, not without
		normalization).
\end{description}
