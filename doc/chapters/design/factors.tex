\section{Factors}\label{sec:factors}

We have identified the following factors that may affect the performance of the
system:
\begin{description}
	\item[Floorplan area] (\(A\)) The area of the floorplan. We expect this
		to affect the total broadcast time (more area to cover; higher
		broadcast time). In the following, we will consider this as a
		fixed parameter, since we make the assumption that the area of
		the floorplan is ``given by the problem'' (we have to cover a
		given area with an optimized network). So we will not consider
		the floorplan area as a tunable factor.
	\item[Floorplan X/Y ratio] (\(\frac{X}{Y}\)) The ratio between the two
		dimensions of the floorplan. This makes the floorplan more
		crushed rectangle when \(\frac{X}{Y} \neq 1\). While when
		\(\frac{X}{Y} = 1\) the floorplan is a square. It may affect all
		the performance indexes defined in \secref{sec:indexes}. For the
		same considerations above, we will not consider this a tunable
		factor.
	\item[Users density] (\(\frac{N}{A}\)) We expect that each performance
		index increases with the number of users in the floorplan. We
		will not consider this a tunable factor: in a real world
		scenario we may have a given number of users (even variable over
		time, but not chosen when the network is installed) to cover.
		Anyway, we will still analyze how the performances of the
		network change when the user density is changed.
	\item[Broadcast radius] (\(R\)) We expect this factor to affect all the
		performance indexes defined in \secref{sec:indexes}, in
		particular the energy efficiency and the total number of
		collisions. Also the percentage of covered users may be highly
		affected by the broadcast radius (indirectly, due to the high
		number of collisions with higher radius; and directly, due to
		the lower number of reached users with low values of \(R\)). Of
		course, the broadcast time is affected since an higher \(R\)
		\emph{may} reduce the broadcast time (if collisions are not
		increased too much).
		This factor may be tuned to improve the performance of the
		network.
	\item[Hear window] (\(T\)) The hear window may affect the total
		broadcast time, the total number of collisions and, combined
		with the maximum copies factor, the energy efficiency. This
		factor may be tuned to improve the performance of the network.
	\item[Max relay delay] (\(\max(\delta)\)) We expect this factor to
		affect mainly the total broadcast time and the number of
		collisions.
	\item[Maximum copies] (\(m\)) This factor may mainly affect the total
		number of collisions, the energy efficiency and the total
		broadcast time.
	\item[Position of first user] The position in the 2D floorplan of the
		users that sends the first message may affect the performance of
		the system. This factor will be studied separately in
		\chref{ch:starting-node}. This is not a tunable factor (from
		\chref{ch:specs}: ``Consider a 2D floorplan with \(N\) users
		randomly dropped in it. A random user within the floorplan
		produces a \emph{message} \elision.'').
\end{description}
