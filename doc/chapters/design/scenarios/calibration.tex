\subsection{Factors calibration}\label{subsec:calibration}

We do not want to have too low values for the \(T\) parameter, since they would
not allow the \emph{trickle relaying} algorithm to work properly. We will vary
this parameter from \(5s\) to \(10s\).

\(\max(\delta)\) can not be too low, or it becomes useless. We will vary this
parameter from \(5s\) to \(10s\).

Regarding the parameter \(m\), we will find out appropriate values for each
scenario by inspecting the \textit{copies} collected statistic. Generally, we
expect that values in the range 2--7 are ok.

The broadcast radius \(R\) must be adapted for every scenario, since the user
density is different:
\begin{enumerate}
	\item For the ``high density scenario'' we will vary \(R\) from \(10m\)
		to \(20m\).
	\item For the ``low density scenario'' we will vary \(R\) from \(30m\)
		to \(50m\).
	\item For the ``rectangular scenario'' we will vary \(R\) from \(30m\)
		to \(50m\).
\end{enumerate}
