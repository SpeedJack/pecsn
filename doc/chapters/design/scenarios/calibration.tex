\subsection{Factors calibration}\label{subsec:calibration}

We do not want to have too low values for the \(T\) parameter, since they would
not allow the \emph{trickle relaying} algorithm to work properly.

For \(\max(\delta)\) we will avoid values greater than \(T\): this parameter
is used to ``de-synchronize'' the time at which the message is sent by different
nodes so it makes no sense to have this delay greater than the hear window.

Regarding the parameter \(m\), we will find out appropriate values for each
scenario by inspecting the \textit{copies} collected statistic. Generally, we
expect that values in the range 3--7 are ok.

The broadcast radius \(R\) must be adapted for every scenario, since the user
density is different, so we have these final calibrations:
\begin{enumerate}
	\item For the ``high density scenario'' we will vary \(T\) from \(5s\)
		to \(10s\), \(\max(\delta)\) from \(5s\) to \(10s\) and \(R\)
		from \(10m\) to \(20m\).
	\item For the ``low density scenario'' we will vary \(T\) from \(2s\)
		to \(5s\), \(\max(\delta)\) from \(2s\) to \(5s\) and \(R\)
		from \(30m\) to \(50m\).
	\item For the ``rectangular scenario'' we will vary \(T\) from \(5s\)
		to \(10s\), \(\max(\delta)\) from \(5s\) to \(10s\) and \(R\)
		from \(20m\) to \(40m\).
\end{enumerate}
