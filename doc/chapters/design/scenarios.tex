\section{Scenarios}\label{sec:scenarios}

In order to ease the analysis, we have defined three scenarios with different
values for \(A\), \(\frac{X}{Y}\) and \(N\). Inside each scenario, we will vary
all other factors and we will evaluate the impact of these factors in the
performances of the network. As said in \secref{sec:factors}, we will also vary
the number of users \(N\) but we will not consider this a tunable factor to
improve the performances of the network.

The scenarios defined are the following:
\begin{enumerate}
	\item \(A = 40000m^2\), \(\frac{X}{Y} = 1\) \idest{\(200m\times200m\)
		square} and \(N = 200\mathit{users}\) \idest{\(\frac{N}{A} =
		\frac{200\mathit{users}}{40000m^2} = 0.005\mathit{users}/m^2 =
		\frac{1}{200}\mathit{users}/m^2\)}. This is the ``low density
		scenario''.
	\item \(A = 10000m^2\), \(\frac{X}{Y} = 1\) \idest{\(100m\times100m\)
		square} and \(N = 500\mathit{users}\) \idest{\(\frac{N}{A} =
		\frac{500\mathit{users}}{10000m^2} = 0.05\mathit{users}/m^2 =
		\frac{1}{20}\mathit{users}/m^2\)}. This is the ``high density
		scenario''.
	\item \(A = 14700m^2\), \(\frac{X}{Y} = 3\) \idest{\(210m\times70m\)
		square} and \(N = 70\mathit{users}\) \idest{\(\frac{N}{A} =
		\frac{70\mathit{users}}{14700m^2} =
		\frac{1}{210}\mathit{users}/m^2 \simeq
		0.00476\mathit{users}/m^2\)}.  This is the ``rectangular
		scenario''.
\end{enumerate}
