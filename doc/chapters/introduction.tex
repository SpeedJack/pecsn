\chapter{Introduction}\label{ch:introduction}

\begin{center}
	\Large \textbf{Epidemic Broadcast}\\
	\normalsize \textit{(trickle relaying)}
\end{center}

Consider a 2D floorplan with \(N\) users randomly dropped in it. A random user
within the floorplan produces a \emph{message}, which should ideally reach all
the users as soon as possible. Communications are \emph{slotted}, meaning that
on each slot a user may or may not relay the message, and a message occupies an
entire slot. A \emph{broadcast radius} \(R\) is defined, so that every receiver
who is within a radius \(R\) from the transmitter will receive the message, and
no other user will hear it. A user that receives more than one message in the
same slot will not be able to decode it (\emph{collision}).

Users relay the message they receive \emph{once}, according to the following
policy (\emph{trickle relaying}): after the user successfully receives a
message, it waits for a time window of \(T\) slots. If during that time window
it correctly receives less than \(m\) copies of the same message, it relays it,
otherwise it stops.

A sender does not know (or cares about) whether or not its message has been
received by its neighbors.

Measure at least the broadcast time for a message in the entire floorplan, the
percentage of covered users, the number of collisions.

In all cases, it is up to the team to calibrate the scenarios so that meaningful
results are obtained.

\begin{tcolorbox}[title=Note]
	We have made a correction to the above specifications in order to fix a
	problem with trickle relaying.

	As defined above, all neighbors users of a sender receive the message in
	the same slot and then they all waits for the same time window of \(T\)
	slots before relaying the message. This leads to an issue where all the
	users relay the message in time slots multiple of \(T\) while the entire
	network stays silent in all other slots.

	In order to fix this issue, we have added a \emph{random relay delay}
	(\(\delta\)) to the time window so that each user waits for \(T+\delta\)
	slots before relaying the message, where \(\delta\) changes every time
	for every user.
\end{tcolorbox}
